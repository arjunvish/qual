\documentclass{article}
\usepackage{proof}
\usepackage{bussproofs}
\begin{document}
\title{Lazy Proofs for DPLL(T)-Based SMT Solvers}
\date{}
\maketitle
\section{Introduction}
SMT (satisfiability modulo theory) solvers input typically large 
formulas that contain both Boolean Logic and logic in different 
theories, and tell you whether the formulas are satisfiable or 
unsatisfiable. \\
Verification tools use these solvers to prove system properties. 
As a result, solver output must be trustable. \\
However, SMT solvers are really complicated tools that have tens 
of thousands of lines of code. \\
One solution to this is to have the SMT solver dispatch a 
certificate of proof of its result. \\
For satisfied formulas, this can be a model of satisfaction, 
that is, values for all the variables in the formula. \\
For unsatisfied formulas, it is a transformation of the formula
into a siimple contradiction using a small set of inference rules, 
checkable by an external tool such as a proof checker. \\
\\
Modern SMT solvers have a SAT solver to reason about Boolean 
logic, and multiple theory solvers for reasoning in each of the 
theories used. \\
All these communicate with each other in subtle ways and 
this must be accounted for while producing proofs. \\
Proofs must also be fine-grained, that is, they must be 
detailed enough so they can be checked by simple means. \\
\\
Contributions made in this paper:
\begin{enumerate}
	\item They formalize a \textit{generalized} approach
	for fine-grained proof generation in DPLL(T)-style
	SMT solvers.
	\item They present a lazy approach of proof generation
	that incurs a low overhead. While deriving a contradiction,
	the solver generates many lemmas, most of which aren't 
	ultimately used to derive the contradiction.
	The proof generation only begins after the contradiction
	is derived, so these unnecessary lemmas don't need to be
	accounted for.
	\item They present a lazy proof generation technique for
	the specific theories of uninterpreted functions with
	equality and the extensional theory of arrays.
\end{enumerate}

The technique was implemented in CVC4 and tested on the SMT-Lib
benchmark library.

\section{DPLL(T)-Based SMT Solvers}
SMT is the problem of determining the satisfiability of a 
set of formulas in some background theory T. \\
This work focuses on quantifier-free formulas and on 
DPLL(T) architecture solvers which combine a SAT engine and 
multiple theory solvers.

\subsection{Abstract DPLL(T) Framework}
The background theory T consists of m theories $T_1,...,T_m$
with respective many-sorted signatures $\Sigma_1,..,\Sigma_m$. \\
A signature is defined as consisting of:
\begin{itemize}
	\item sort symbols.
	\item predicate symbols with an associated arity.
	\item function symbols with an associated arity.
\end{itemize}

In this setting, all signatures share the set of sort symbols 
$\textbf{S}$, and equality is the only predicate. \\
The theories share no function symbols except for a set 
$C = \cup_{S \in \textbf{S}}\ C_S$, where each $C_S$ is a 
distinguished infinite set of free/uninterpreted constants
of sort $S$. Thus, the only elements that distinguish the 
signatures are the constant symbols that don't belong to 
$C$ and the non-nullary function symbols. \\ \\
DPLL(T) solvers can be formalized abstractly as 
state transition systems defined by a set of transition rules.\\
The states of a system are either
\begin{itemize}
	\item $fail$
	\item $\langle M, F, C \rangle$
	where $M$ is the current context; $F$ is a set of ground clauses
	representing some form of the input formula; $C$ represents the
	conflict clause which may be the empty set when there isn't
	a conflict.
\end{itemize}
If $M = M_0 \bullet M_1 \bullet ... \bullet M_n$, each $M_i$ is the
decision level, and $M^{[i]}$ denotes $M_0 \bullet ... \bullet M_i$. \\
Initial state : $\langle \phi, F_0, \phi \rangle$, where $F_0$ is 
the input formula. \\
Final state:
\begin{itemize}
	\item $fail$, when $F_0$ is unsatisfiable in T.
	\item $\langle M, F, \Phi \rangle$ where $M$ is satisfiable in
	T, $F$ is equisatisfiable with $F_0$ in T, and $M \models_P F$.
\end{itemize}

Each atom of a clause $F \cup C$ is pure, that is, it has a signature
$\Sigma_i$ for some $i \in {1,...,m}$. \\
$Int_M$ is the set of all \textit{interface literals} of M:
the (dis)equalities between shared constants. \\
\textit{Shared constants} are the set represented by \\
$\{c | constant\ c\ occurs\ in\ Lit_{M|i} and Lit_{M|j}, for
some 1 \leq i < j \leq m\}$. \\
$Lit_{M|i}$ consists of the $\Sigma_i$- literals of $Lit_M$. \\
In other words, $Int_M$ is the set of (dis)equalities between
elements of $C$.\\
\\
Transition rules:
\begin{itemize}
	\item Propagate:\\ $\infer[Prop]{M := Ml}{l_1 \lor ... \lor l_n \lor l \in F 
	& \neg l_1,...,\neg l_n \in M & l, \neg l \notin M}$ \\
	If all but one literal in a clause have been negated in the 
	current context, positively assigned the last literal to satisfy
	the clause.
	\item Decide:\\ $\infer[Dec]{M := M \bullet l}
	{l \in Lit_F \cup Int_M & l, \neg l \notin M}$ \\
	Pick an unassigned literal either from the formula $F$, or
	from the interface literals and assign it positively.
	\item Conflict:\\ $\infer[Confl]{C:=\{l_1 \lor ... \lor l_N\}}
	{C = \phi & l_1 \lor ... \lor l_n \in F & \neg l_1,...,\neg l_n \in M}$ \\
	If there is a clause in F, all of whose literals have been negated by
	the current context, then that clause is the conflict clause.
	\item Explain:\\ $\infer[Expl]{C := \{l_1 \lor ... \lor l_n \lor D\}}
	{C = \{\neg l \lor D\} & l_1 \lor ... \lor l_n \lor l \in F 
		& \neg l_1,...,\neg l_n \prec_M l}$ \\
	If there is a clause in F that contains n literals that were 
	assigned in the context before the literal l was assigned,
	such that l occurs in the conflict clause, resolve these two 
	clauses to get the new conflict clause.
	\item Backjump:\\ $\infer[Backj]{C := \phi\ \ \ M := M^{[i]}l}
	{C = \{l_1 \lor ... \lor l_n \lor l\} & 
		lev\ \neg l_1, ... , lev\ \neg l_n \leq i < lev\ \neg l}$ \\
	If one literal in the conflict clause is in a level strictly above 
	the levels of all other literals in M, then rewind the context to any
	level in between these two levels.
	\item Learn: \\ $\infer[Learn]{F := F \cup C}{C \neq \phi}$ \\
	If there is a conflict clause, it is entailed by input formula, 
	so add it to the set of clauses to satisfy $F$.
	\item Fail:\\ $\infer[Fail]{fail}{C \neq \phi & \bullet \notin M}$ \\
	If there is a conflict, and there are no decision points to 
	backjump on, then move to the fail state, that is, the input 
	formula is unsatisfiable.
\end{itemize}
The rules above model the behavior of the SAT engine, which treats atoms
as Boolean variables. The rules that follow model the interaction between 
the SAT solver and the theory solvers.
\begin{itemize}
	\item $Propagate_i$ \\ $\infer[Prop_i]{M := Ml}	{l \in Lit_F \cup Int_M & 
		\models_i l_1 \lor ... \lor 
		l_n \lor l & \neg l_1, ..., \neg l_n \in M & l,\neg l \notin M}$ \\
	If all but one literal of a theory-i lemma are falsified by the 
	current context, and the one literal is unassigned and also belongs to 
	either the literals in F or the interface literals in M, then
	positively assign the one literal.
	\item $Conflict_i$ \\ $\infer[Confl_i]{C := \{l_1 \lor ... \lor l_n \}}
	{C = \phi & \models_i l_1 \lor ... \lor l_n & 
		\neg l_1, ..., \neg l_n \in M}$ \\
	If all the literals of a theory-i lemma are falsified by the 
	current context, then the lemma is a conflict clause.
	\item $Explain_i$ \\ $\infer[Expl_i]{C := \{l_1 \lor ... \lor l_n \lor D\}}
	{C = \{\neg l \lor D\} & \models_i l_1 \lor ... \lor l_n \lor l & 
		\neg\ l_1, ..., \neg\ l_n \prec_M l}$ \\
	If one of the literals $l$ in the conflict clause occurs in a 
	theory-i lemma such that the negations of all other literals in 
	that lemma were assigned in the current context before l was  
	assigned positive, then resolve the conflict clause and the 
	theory lemma.
	\item $Learn_i$ \\ $\infer[Learn_i]{F := F \cup \{l_1[\textbf{c}] 
		\lor ... \lor l_n[\textbf{c}]\}}
	{l_1,...,l_n \in Lit_{M|i} \cup Int_M \cup L_i & 
		\models_i \exists \textbf{x}(l_1[\textbf{x}] \lor ... 
		\lor l_n[\textbf{x}])}$ \\ where $\textbf{x}$ is a possibly empty 
	tuple of variables, and $\textbf{c}$ is a tuple of fresh constants 
	from $C$ - the set of shared constants between the sorts - of the
	same sort as $\textbf{x}$; $L_i$ is a finite set consisting of literals
	not present in the original formula $F$. \\
	If a theory-i lemma is valid for some tuple $\textbf{x}$ of variables,
	instantiate it with a corresponding tuple of constants $\textbf{c}$,
	and add it to F.
\end{itemize}
The rules maintain the invariant that every conflict clause and learned 
clause is entailed in T by the initial clause set. \\

\subsection{Modeling Solver Behavior}
Generally speaking, the system uses the SAT engine - rules $Propagate$, 
$Decide$, $Conflict$, $Explain$, $Backjump$, $Learn$, and $Fail$ - to 
construct the context $M$ as a truth assignment for the clauses in $F$, 
as if those clauses were propositional. However, it periodically asks 
the solver of each theory $T_i$
\begin{itemize}
	\item to check if the set of $\Sigma_i$- constraints in $M$ is 
	unsatisfiable in $T_i$ using the $Conflict_i$ rule, or
	\item to check if the set of $\Sigma_i$- constraints in M entails
	yet-undetermined literal from $Lit_F \cup Int_M$ using the 
	$Propagate_i$ or the $Explain_i$ rule.
\end{itemize}
We assume that each $T_i$- solver provides an ${\tt explain_i}$ method
with the property that if $l$ is a literal propagated by the solver, 
then ${\tt explain_i}(l)$ returns a subset $\{\neg l_1, \neg l_2, 
..., \neg l_n\}$ of $M$, such that $\models_i l_1 \lor 
... \lor l_n \lor l$. This way, we can use the $Explain_i$ rule on
literals that have been propagated by theory lemmas, to rewind 
theory propagations and get to decisions.\\
For $i = 1,...,m$ the set $Lit_{M|i}$ consists of the $\Sigma_i$- 
literals of $Lit_M$. \\

The set $Lit_F$ (resp., $Lit_M$) consist of all literals in F 
(resp., in M) and their complements. For $i = 1,...,m$, the 
set $L_{M|i}$ consists of the $\Sigma_i$-literals of $Lit_M$.
$Int_M$ is the set of all $\textit{interface literals}$ of M: 
the (dis)equalities between $\textit{shared constants}$, where
the set of shared constants is  \\ $\{c$ $|$ constant $c$ occurs in 
$Lit_{M|i}$, and $Lit_{M|j}$, for some $1 \leq i < j \leq m\}$.
The inclusion of $Int_M$ in rules $Decide$ and $Propagate_i$ 
achieves the effect of the Nelson-Oppen theory combination method.

For example, consider the following formula. \\
F : $1 \leq x \land x \leq 2 \land f(x) \neq f(2) \land x = 1$.
The theories involved in this formula are the theory of equality
over uninterpreted functions (E) and the theory of linear integer
arithmetic (Z). Literals here aren't pure, that is, they belong to 
more than a single theory. To change that, variables are abstracted: \\
F : $a \leq x \land x \leq b \land f(x) \neq f(b) \land x = a 
\land a = 1 \land b = 2$. \\
$Lit_F = \{a \leq x ,\ x \leq b ,\ f(x) \neq f(b) ,\ x = a 
,\ a = 1 ,\ b = 2\}$ \\
$L_{F|E} = \{f(x) \neq f(b),\ x = a\}$ \\
$L_{F|Z} = \{a \leq x ,\ x \leq b ,\ x = a,\ a = 1 ,\ b = 2\}$\\
$Shared\ constants = \{a,\ x,\ b\}$\\
$Int_F = \{x = a\}$\\

Rule $Learn_i$ models theory solvers following the 
splitting-on-demand paradigm. When asked about the satisfiability of 
the set of $\Sigma_i$-literals in M, such solvers may return instead 
a $\textit{splitting lemma}$ - a clause encoding a guess that needs 
to be made about those literals before the solver can determine 
their satisfiability. The set $L_i$ is a finite set consisting of 
additional literals, that is, not present in the original formula 
in F, which may be generated by splitting-on-demand theory solvers.\\
For example, when the array solver is asked whether the following is
satisfiable:\\
$read(write(A,i,v),j) = read(A,j)$\\
it returns the following splitting lemma:\\
$read(write(A,i,v),j) \neq read(A,j) \lor i \neq j 
\lor read(A,i) = v$.\\
In other words, the equation $read(write(A,i,v),j)=read(A,j)$ holds
in 2 situations: when $i$ and $j$ are distinct, or when they are
equal but $write(A,i,v)$ changes nothing.

\section{Generating Proofs in DPLL(T)}
The transition rules defined above are refutation sound: if an 
execution starting with $\langle \phi, F_0, \phi \rangle$ ends with 
$fail$, then $F_0$ is unsatisfiable in $T$.

\subsection{Proof Generation for Propositional Unsatisfiability}
A failed execution can be understood as trying to construct a 
refutation tree: a tree of clauses built from the leaves,
which are either clauses in $F_0$ or learned clauses, down to
the root $\bot$, where each non-leaf node is a 
propositional resolvent of its children. Additionally, one
needs to prove that each learned clause is a consequence of 
the initial clause set.

For example, consider the formula F in CNF expressed as the set:\\
$\{a \lor \neg b, \neg a \lor \neg b, b \lor c, \neg c \lor b\}$
\begin{center}
	\begin{tabular}{l l l l l}
		\textbf{M} & \textbf{F} & \textbf{C} & \textbf{Rule} & \textbf{Step}\\
		\hline
		& $a \lor \neg b, \neg a \lor \neg b, b \lor c, \neg c \lor b$ & $\phi$ & Dec & 1 \\
		$\bullet a$ & $a \lor \neg b, \neg a \lor \neg b, b \lor c, \neg c \lor b$ & $\phi$ & Prop ($a \lor \neg b$) & 2 \\
		$\bullet a \neg b$ & $a \lor \neg b, \neg a \lor \neg b, b \lor c, \neg c \lor b$ & $\phi$ & Prop $(b \lor c)$ & 3 \\
		$\bullet a \neg b\ c$ & $a \lor \neg b, \neg a \lor \neg b, b \lor c, \neg c \lor b$ & $\phi$ & Confl $(\neg c \lor b)$ & 4 \\
		$\bullet a \neg b\ c$ & $a \lor \neg b, \neg a \lor \neg b, b \lor c, \neg c \lor b$ & $\neg c \lor b$ & Expl $(b \lor c)$ & 5 \\
		$\bullet a \neg b\ c$ & $a \lor \neg b, \neg a \lor \neg b, b \lor c, \neg c \lor b$ & $b$ & Expl $(\neg a \lor \neg b)$ & 6 \\
		$\bullet a \neg b\ c$ & $a \lor \neg b, \neg a \lor \neg b, b \lor c, \neg c \lor b$ & $\neg a$ & Learn $(\neg a)$ & 7\\
		$\bullet a \neg b\ c$ & $a \lor \neg b, \neg a \lor \neg b, b \lor c, \neg c \lor b, \neg a$ & $\neg a$ & Backj $(\neg a)$ & 8 \\
		$\neg a$ & $a \lor \neg b, \neg a \lor \neg b, b \lor c, \neg c \lor b, \neg a$ & $\phi$ & Prop $(a \lor \neg b)$ & 9 \\
		$\neg a \neg b$ & $a \lor \neg b, \neg a \lor \neg b, b \lor c, \neg c \lor b, \neg a$ & $\phi$ & Prop $(b \lor c)$ & 10 \\
		$\neg a \neg b\ c$ & $a \lor \neg b, \neg a \lor \neg b, b \lor c, \neg c \lor b, \neg a$ & $\phi$ & Confl $(\neg c \lor b)$ & 11 \\
		$\neg a \neg b\ c$ & $a \lor \neg b, \neg a \lor \neg b, b \lor c, \neg c \lor b, \neg a$ & $\neg c \lor b$ & Expl $(b \lor c)$ & 12 \\
		$\neg a \neg b\ c$ & $a \lor \neg b, \neg a \lor \neg b, b \lor c, \neg c \lor b, \neg a$ & $b$ & Expl $(a \lor \neg b)$ & 13\\
		$\neg a \neg b\ c$ & $a \lor \neg b, \neg a \lor \neg b, b \lor c, \neg c \lor b, \neg a$ & $a$ & Expl $(\neg a)$ & 14 \\
		$\neg a \neg b\ c$ & $a \lor \neg b, \neg a \lor \neg b, b \lor c, \neg c \lor b, \neg a$ & $\bot$ & Fail & 15 \\
	\end{tabular}
\end{center}
The following is a proof tree for the unsatisfiability of F.
\begin{prooftree}
	\AxiomC{$\neg c \lor b$ (4)}
	\AxiomC{$b \lor c$ (5)}
	\BinaryInfC{$b$}
	\AxiomC{$\neg a \lor \neg b$ (6)}
	\BinaryInfC{$\neg a$ (14)}
	\AxiomC{$\neg c \lor b$ (11)}
	\AxiomC{$b \lor c$ (12)}
	\BinaryInfC{$b$}
	\AxiomC{$a \lor \neg b$ (13)}
	\BinaryInfC{$a$}
	\BinaryInfC{$\bot$}
\end{prooftree}
Steps 1 to 7 constitute the left subtree of the proof tree that
adds $\neg a$ as a learned clause to F. Steps 8 to 14 constitute 
the left subtree of the proof tree.
Conflict clauses and explanations represent propositional resolution
between clauses and the proof tree has a node for every application 
of the Conflict and the Explain rules, and one at the root for the 
Fail rule.

\subsection{Proof Generation for Unsatisfiability Modulo Theories}
The non-propositional rules are can also be seen as trying to
construct a refutation tree. However, in addition to clauses from 
the initial formula, and clauses that are propositionally learned,
leaves can have additional clauses that are valid in the particular
theory - these are called \textit{theory lemmas}. \\
The rules $Conf_i$, $Learn_i$, and $Expl_i$ add theory lemmas 
to the refutation proof tree, while $Prop_i$ uses theory lemmas to 
propagate literals. \\
The theory lemmas added to the proof tree must be proven, and this 
is done by the corresponding theory solver. Each $T_i$-solver 
provides a method $\tt{provideProof_i}$ that takes as input 
a theory lemma and returns a proof of that lemma using theory 
specific proof rules.  

\end{document}