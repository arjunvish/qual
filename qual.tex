\documentclass{article}
\usepackage{proof}
\usepackage{bussproofs}
\usepackage{xcolor}
\usepackage{url}
\usepackage{framed}
\usepackage{amsmath, amsfonts, amssymb}
\usepackage{mathtools}

\begin{document}
\begin{titlepage}
	\centering
	{\textbf{\Large University of Iowa \\ Computer Science Department}}
	\vspace{1cm}
	
	{\Huge Comparison of Proof Producing Systems in SMT Solvers \par}
	\vspace{1cm}
	
	{\large SUBMITTED IN PARTIAL FULFILLMENT OF THE REQUIREMENTS 
		FOR THE QUALIFYING EXAM OF PH.D. PROGRAM \par}
	\vspace{2cm}
	
	{\Large Arjun Viswanathan \par}
	{\large University of Iowa - Computer Science Department \par}
	{\large arjun-viswanthan@uiowa.edu}
	\vspace{2cm}
	
	{\large Advisor \par}
	{\Large Cesare Tinelli \par}
	{\large cesare-tinelli@uiowa.edu \par}
	
	\vfill
	{\large Fall 2018}
\end{titlepage}

\title{Comparison of Proof Producing Systems in SMT Solvers}
\date{}
\maketitle
\begin{abstract}
Satisfiability Modulo Theories (SMT) solvers input typically
large formulas that contain both Boolean logic and logic in 
different theories - such as arithmetic and strings - and 
decide whether 
the formulas are satisfiable or unsatisfiable. Verification 
tools use these solvers to prove system properties. As a 
result, solver output must be trustable. However, SMT solvers 
are really complicated tools that have tens of thousands of 
lines of code. One way to make SMT solver ouput more reliable is 
to have them produce proofs of their results. 
For satisfied formulas, this can be a model of 
satisfaction, that is, values for all the variables in the
formula. For unsatisfied formulas, it is a transformation of 
the formula into a simple contradiction using a small set of 
inference rules, checkable by an external tool such as a proof 
checker. This report describes the proof producing mechanisms 
commonly used in SMT solvers and compares the proof systems 
of CVC4, VeriT, and Z3, three state-of-the-art SMT solvers.
\end{abstract}

\section{Introduction}
\label{sec:intro}
Boolean satisfiability, often called the SAT problem, 
is the problem of satisfying a Boolean formula, that is, 
consistently assigning values of $True$ or $False$ to the variables 
of the formula so that the entire formula evaluates to $True$. 
For example,
\begin{center}$(x \lor y) \land z$ \end{center}
can be satisfied by the 
assignment $\{x=True,\ y=False,\ z=True\}$. On the other hand, 
\begin{center} $x \land \neg x$ \end{center}
is unsatisfiable, no matter what value is assigned to $x$.

Satisfiability Modulo Theories~\cite{DBLP:reference/mc/BarrettT18} 
or SMT lifts SAT to a level that includes theories. 
For example, 
\begin{center} $(a = b) \land (b = c) \land \neg (a = c)$ 
\end{center}
is a formula that is unsatisfiable in the theory of 
equality over uninterpreted functions. This is because, by
transitivity of $a = b$ and $b = c$, we have $a = c$. SMT 
allows us to be more expressive with our formulas, but 
this comes at the cost of more complicated decision 
procedures.

SMT solvers have plenty of applications in formal methods 
and software verification. For instance, SMT solvers are used 
in the back-end of model checkers~\cite{DBLP:books/daglib/0020348}, 
which input mathematical 
models of a software system, and verify whether they 
satisfy a particular property or not. Another area of 
application is symbolic
execution~\cite{DBLP:journals/csur/BaldoniCDDF18}, 
which is to analyze a 
program to figure out what set of inputs work for each 
part of the program. Other uses of SMT solvers include 
program synthesis~\cite{synth}, static analysis, 
and interpolant generation~\cite{DBLP:journals/corr/abs-1111-5652}.

Given their rise in popularity and usage in the software 
verification world, it is really important that we are able 
to trust the outputs of SMT solvers. SMT solvers are typically 
very complex systems with tens of thousands of lines of code, 
likely to contain bugs. Verifying such a large codebase can 
be a cumbersome, if not impossible task. An alternative is 
to rely on tools called proof checkers~\cite{proofasst}, 
that have a much more trusted kernel containing a small 
set of axioms and inference rules. These proof checkers 
sacrifice in automative capability what they gain in terms 
of trustability over SMT solvers. To exploit these proof 
checkers, SMT solvers produce proof certificates of their 
outputs, that can be checked by the proof checker.

In this report, I introduce the workings of an SMT solver 
and their proof producing capabilities; I also compare the 
proof systems of CVC4, VeriT, and Z3 - three 
state-of-the-art SMT solvers - as respectively 
described in the research papers titled ``Lazy Proofs for 
DPLL(T)-Based SMT Solvers"~\cite{DBLP:conf/fmcad/KatzBTRH16}, 
``Expressiveness + Automation + Soundness : 
Towards Combining SMT Solvers and Interactive 
Proof Assistants"~\cite{DBLP:conf/tacas/FontaineMMNT06}, 
and ``Proofs and Refutations, and Z3"~\cite{DBLP:conf/lpar/MouraB08}.
I begin by explaining the algorithm that is at the core 
of most SMT solvers. I then explain the logical basis for 
proof production in SMT solvers, and then get into 
comparing the research work as I explain the proof 
production systems of the SMT solvers.


\section{Formal Preliminaries}
\label{sec:prelim}
The following grammar specifies the syntax of the formulas we
will use: 
\begin{align*}
&Variable &\rightarrow\  &x\ |\ y\ |\ ...  \\
&Constant &\rightarrow\ &c_1\ |\ c_2\ |\ ...  \\
&Term &\rightarrow\ &Variable\ |\ Constant\ |\ f(t_1, ..., t_n) \\
&Literal &\rightarrow\ &Term\ |\ \neg Term \\
&Formula &\rightarrow\ &True\ |\ False\ |\ p(t_1, ..., t_n)\ 
					|\ Literal\ |\ Literal \lor Literal \\ 
& & &				|\ Literal \land Literal\ 
					|\ Literal \Rightarrow Literal\ 
					|\ Literal \iff Literal \\
\end{align*}
where $f$ is a function symbol and $p$ is a predicate symbol.

Functions can be understood as they are in basic mathematics, 
and predicates are functions to the Boolean type. Negation 
($\neg$), disjunction ($\lor$), conjunction ($\land$), 
implication ($\Rightarrow$), and double implication
($\iff$) have semantics as in classical logic~\cite{prop}.

Our setting consists of a background theory T consisting 
of m theories $T_1, ..., T_m$ with respective many-sorted 
signatures $\Sigma_1, ..., \Sigma_m$. All signatures share 
a set of sort (type) symbols, and equality is the only 
predicate. The theories also share a set of 
uninterpreted constants which are used for reasoning about 
terms that belong to multiple theories. 

The preceding part of this section formalizes the fact 
that the framework here is 
described for a single theory, but that theory can 
be considered a combination of multiple 
theories~\cite{Manna2003}. This means that formulas are 
composed of Boolean components, each of which will 
evaluate to $True$ or $False$. However, since 
we have theories, these components could be just Boolean 
variables, or they could be (dis)equalities over our 
multiple theories, which gives us expressivity. 

For example, the following formula is over 
the theory of linear integer 
arithmetic (LIA) and the theory of 
equality over uninterpreted functions (EUF)
~\cite{LIAEUF}.\\
$(x \land y) \Rightarrow (a = 0 \land a + b = c \land 
\neg (f(b) = f(c)))$ \\
$x$ and $y$ are terms in the 
formula, and hence can only evaluate to $True$ or $False$.
$a$, $b$, and $c$, on the other hand, are integer variables 
that must evaluate to an integer constant. To generalize 
this, we assume our theory T to be a combination of $m$ 
theories $T_1, ..., T_m$. Each theory $T_i$ is formally 
described by means of a signature $\Sigma_i$, but we 
will stick to intuitive theories such as the ones mentioned 
before, so we don't have to deal with such formalisms. A 
formal definition of theories and SMT can be found 
at~\cite{DBLP:reference/mc/BarrettT18}.

When it is used as a set, $\phi$ refers to the empty set. 
We also allow for representation of formulas that are 
conjunctions, as sets of the corresponding 
conjuncts. So we represent $c_1 \land ... \land c_n$ as
$\{c_1, ..., c_n\}$.

We need the notion of entailment on two levels. Propositional 
entailment: \\
$p \models_P q$ \\
read as ``formula $p$ propositionally entails formula $q$" says 
that $q$ is a logical consequence of $p$. For example, 
$a \land b \models_P a$, where a and b are Boolean 
variables. Entailment also occurs in a theory: \\
$x \models_i y$ \\
read as ``formula $x$ $i$-entails formula $y$", where 
$i$ is the concerned theory. For example, 
$x > 3 \models_{LIA} x > 0$, where x is an integer variable.
If we abstracted these formulas to the propositional level, 
$m = (x > 3)$, and $n = (x > 0)$, 
the entailment $m \models_P n$ cannot be realized at the 
propositional level. We need reasoning at the level of 
the theory of arithmetic for this entailment to hold.

A formula is $satisfiable$ if we can consistently assign
values to all its variables (Boolean and theory), 
so that the formula evaluates to $True$. A formula is 
$unsatisfiable$ if there is no consistent assignment that 
we can give its variables so that the formula evaluates to 
$True$. $(x + y = 0) \land (m \lor n)$ is satisfiable, 
and a satisfying assignment is $\{x = 0,\ y = 0,\ m = True,\ 
n = False\}$. $p \land \neg p$ is unsatisfiable. Two formulas 
are $equisatisfiable$ if they are both satisfiable, or if they 
are both unsatisfiable. Two formulas are $equivalent$ 
if every model that satisfies 
one of them satisfies the other as well. Consider 
$a \land b$ and $a \land b \land n$. These are 
equisatisfiable since both of them are satisfiable - 
$\{a = True,\ b = True\}$ satisfies the first one and 
$\{a = True,\ b = True,\ n = True\}$ satisfies the second one.
However, they are not equivalent since model 
$\{a = True,\ b = True,\ n = False\}$ satisfies the first one, 
but not the second one.

A formula is $valid$, if no matter what values we assign its 
variables, it evaluates to $True$. In other words, it is
entailed from nothing. For example, \\
$\models_P p \lor \neg p$, and \\
$\models_{LIA} (x = 0) \lor (x < 0) \lor (x > 0)$.

Finally, the universal quantifier ($\forall$) and 
the existential quantifier ($\exists$) help us quantify 
variables in formulas. For example, $\forall x, x > 0 
\Rightarrow \neg (x < 0)$ states the obvious fact that 
positive numbers aren't negative, and $\exists x, y = 2x$ 
is the predicate that is true if $y$ is even. Our discussion 
in this report only involves quantifier-free logic of SMT 
solvers, but these are useful tools to have at the meta-level.


\section{The DPLL(T) Algorithm}
\label{sec:dpll}
Davis-Putnam-Logemann-Loveland or DPLL - named after its 
developers - is an algorithm developed in the 1960s 
for deciding the satisfiability of propositional formulas.
About half a century later, most SAT solvers and SMT solvers
are still based on some form of DPLL. All the solvers 
discussed in this work are DPLL(T) solvers, that is an 
extension of DPLL to accommodate theories. 
Section~\ref{sec:cnf}
introduces the normal form required by these solvers, 
and Section~\ref{sec:trans} explains the DPLL(T) algorithm as a 
transition system as presented 
in~\cite{DBLP:conf/fmcad/KatzBTRH16}.

\subsection{CNF Conversion}
\label{sec:cnf}
\begin{framed}
SMT solvers require that the input formula is converted to a 
normal form called conjunctive normal form (CNF), before they 
start solving them. The CNF of a formula represents the 
formula as a conjunction of disjunctions. A clause is a 
disjunction (Boolean OR) of variables. So CNF is a 
conjunction (Boolean AND) of clauses. 
\end{framed}

Using Tseitin clausification~\cite{BEL01Handbook}, any formula 
can be converted into an equisatisfiable CNF formula.

The following is an example where a set of rewrite rules
are applied to F to convert it into CNF. 
\begin{align*}
	&F&\ &a \Rightarrow (b \land c) \\
	&Step\ 1&\ &a \Rightarrow x_1 
		\land (x_1 \iff (b \land c)) \\
	&Step\ 2&\ &x_2 \land (x_2 \iff a \Rightarrow x_1) 
		\land (x_1 \iff (b \land c)) \\
	&Step\ 3&\ &x_2 \land \\
	& & &(x_2 \Rightarrow (a \Rightarrow x_1) \land 
		(a \Rightarrow x_1) \Rightarrow x_2) \land \\
	& & &(x_1 \Rightarrow (b \land c) \land 
		(b \land c) \Rightarrow x_1) \\
	&Step\ 4&\ &x_2 \land \\
	& & &(\neg x_2 \lor \neg a \lor x_1) \land 
		(a \lor x_2) \land (\neg x_1 \lor x_2) \land \\
	& & &(\neg x_1 \lor b) \land (\neg x_1 \lor c) \land 
		(\neg b \lor \neg c \lor x_1) \\
\end{align*}

Steps 1 and 2 involve introducing fresh variables $x_1$
and $x_2$ for all subterms of F by means of equivalence 
between the subterm and the fresh variable. Steps 3 and 4 
reduce these subterms to CNF by common rewrite rules 
such as distribution and De Morgan's law.

This example is to illustrate that the conversion can be done.
There are many optimizations that are applied in addition, 
that keep the normal form's size in check.


\subsection{Abstract DPLL(T) Framework}
\label{sec:abst}
\begin{framed}
	At a high level, DPLL(T) takes a formula in CNF, and tries 
	to satisfy every conjunct. There are two levels of reasoning 
	- propositional and theory. The algorithm can be understood as 
	always working in the propositional level (thoery literals
	are abstracted to propositional ones), trying to 
	satisfy the formula. If the formula is propositionally 
	unsatisfiable, then the algorithm concludes, without any 
	theory reasoning that the formula is unsatisfiable. However, 
	if the formula is propositionally satisfiable by a model, 
	then the SAT solver sends the satisfying model to the 
	theory solver(s) concerned to check whether it is satisfiable
	at the thoery level as well. If it is satisfiable in the 
	theory level, then the solver returns to the user that 
	the formula is satisfiable. If the theory solver finds an 
	inconsistency, it sends back information to the 
	propositional solver, restricting the current model, 
	and the solver tries again. If the propositional solver 
	is unable to please all the theory solvers 
	for any model that it finds at the propositional level, then
	we conclude by exhaustion that the formula is unsatisfiable.
\end{framed}

\subsubsection{Transition System}
	\label{sec:trans}
DPLL(T) solvers can be formalized abstractly as 
state transition systems defined by a set of transition rules.\\
The states of a system are either
\begin{itemize}
	\item $fail$
	\item $\langle M, F, C \rangle$
\end{itemize}
$M$ is the current context, that is it is the current 
assignment of literals in the formula; a literal in M is 
preceded by a $\bullet$ if the literal was a decision, 
that is, it was guessed. 
If $M = M_0 \bullet M_1 \bullet ... \bullet M_n$, 
each $M_i$ is the decision level, and $M^{[i]}$ 
denotes $M_0 \bullet ... \bullet M_i$. \\
$F$ is a set of clauses representing some form of
the input formula in CNF. $C$ represents the conflict clause,
the clause from $F$ that is falsified by the assignment. \\
Initial state : $\langle \phi, F_0, \phi \rangle$, where $F_0$
is the input formula converted to CNF. \\
Final state:
\begin{itemize}
	\item $fail$, when $F_0$ is unsatisfiable in T.
	\item $\langle M, F, \Phi \rangle$ where $M$ is satisfiable in
	T, $F$ is equisatisfiable with $F_0$ in T, and $M \models_P F$.
\end{itemize}

%Each atom of a clause $F \cup C$ is pure, that is, it has a signature
%$\Sigma_i$ for some $i \in {1,...,m}$.
$Int_M$ is the set of all \textit{interface literals} of M:
the (dis)equalities between shared constants. \\
\textit{Shared constants} are the set represented by \\
$\{c\ |\ constant\ c\ occurs\ in\ Lit_{M|i}\ and\ Lit_{M|j},\ 
for\ some\ 1 \leq i < j \leq m\}$. \\
$Lit_{M|i}$ consists of the $\Sigma_i$- literals of $Lit_M$. \\
These technicalities are necessary for solving of clauses 
that are a combination of multiple theories. 
For example, the term \\
$f(a) = 1 + x$ \\
combines the theory of equality over uninterpreted functions,
and that of arithmetic. Since each theory has its own solver,
this term that uses functions from both theories, must be 
purified for each theory solver, and this is done by means of
a shared constant. If the term is replaced by the following 
terms, \\
$f(a) = s_1, s_1 = 1 + x$ \\
we now have one term that is entirely over uninterpreted 
functions, and one over arithmetic ones. \\

The rules of the transition systems can be split into a 
set of propositional rules, and a set of theory rules. 

\subsubsection{Propositional Rules}
	\label{sec:proprules}
	
\begin{figure}[t]
	\centering
	\begin{framed}
		\begin{tabular}{c}
			$\infer[Prop]{M := Ml}{l_1 \lor ... \lor l_n \lor l \in F 
				& \neg l_1,...,\neg l_n \in M & l, \neg l \notin M}$ \\ \\
			$\infer[Dec]{M := M \bullet l}
			{l \in Lit_F \cup Int_M & l, \neg l \notin M}$ \\ \\
			$\infer[Confl]{C:=\{l_1 \lor ... \lor l_n\}}
			{C = \phi & l_1 \lor ... \lor l_n \in F & \neg l_1,...,\neg l_n \in M}$ \\ \\
			$\infer[Expl]{C := \{l_1 \lor ... \lor l_n \lor D\}}
			{C = \{\neg l \lor D\} & l_1 \lor ... \lor l_n \lor l \in F 
				& \neg l_1,...,\neg l_n \prec_M l}$ \\ \\
			$\infer[Backj]{C := \phi\ \ \ M := M^{[i]}l}
			{C = \{l_1 \lor ... \lor l_n \lor l\} & 
				lev\ \neg l_1, ... , lev\ \neg l_n \leq i < lev\ \neg l}$ \\ \\
			$\infer[Learn]{F := F \cup C}{C \neq \phi}$ \\ \\
			$\infer[Fail]{fail}{C \neq \phi & \bullet \notin M}$ \\ \\
		\end{tabular}
	\end{framed}
	\caption{Transition rules at Propositional level}
	\label{fig:proprules}
\end{figure}

Figure~\ref{fig:proprules} enumerates the propositional rules.
These rules constitute the DPLL(T) algorithm at the
propositional level, or just DPLL.
They model the behavior of the SAT engine, 
which treats atoms
as Boolean variables. Propagations (Prop) allow us to assign 
literals that we are forced to assign by the logic. 
For example, if the input formula is $\neg a \land (a \lor b)$, 
then $a$ must be assigned to $False$, so that the $\neg a$ 
conjunct evaluates to $True$. As a consequence, of this, 
$b$ must be $True$ so that the next conjunct is $True$. Both 
of these assignments are propagations that the logic forces 
us to assign, given that we are trying to satisfy the formula.
We might not always have this luxury; sometimes, we may have 
to make a guess on an assignment, and decisions (Dec) let us do 
this. If we are trying to satisfy 
$(\neg a \lor c) \land (\neg b \lor d) \land (a \lor b)$,
each conjunct is a 
formula, so we need to guess the value of one of the literals. 
Propogations and/or decisions could lead us to 
a point where our current assignment conflicts with 
our goal of trying to satisfy the input formula. The 
conflict rule (Confl) recognizes this. In the previous 
example, assume we came up with the assignment 
$\{a = False,\ b = False\}$ by decisions. This assignment 
satisfies the first two clauses, but falsifies the third, 
so the third clause 
$a \lor b$ is recognized as a conflict clause.
If we encounter a conflict, 
and there were no previous decisions made, then we were 
forced by the logic to arrive at that conflict, so we 
conclude that the input formula is unsatisfiable. The Fail 
rule ensures this. If, there are previous decisions at
the time of conflict, then the explain rule (Expl) 
and the backjump rule (Backj) work together to rewind 
the state to a point where the decision is flipped. 
Decisions cause propagations, and once we suspect that 
a decision was a wrong decision, we want to undo the 
propagations that were caused by it. This is intuitively 
what the explain rule does. Once this is done, the backjump rule is able to flip the decision. A conflict clause always 
consists of a formula that is entailed by the input formula. 
Explanations could transform the conflict 
clause into a clause that we didn't have to begin with, 
and this could be useful information for future propagations. 
The Learn rule allows us 
to add a non-empty conflict clause to the input clauses to be 
satisfied. 


\subsubsection{DPLL Example}
	\label{sec:propex}
\begin{figure}[t]
	\begin{center}
		\begin{tabular}{l l l l l}
			\textbf{M} & \textbf{F} & \textbf{C} & \textbf{Rule} & \textbf{Step}\\
			\hline
			& $a \lor \neg b, \neg a \lor \neg b, b \lor c, \neg c \lor b$ 
			& $\phi$ & Dec & 1 \\
			$\bullet a$ & $a \lor \neg b, \neg a \lor \neg b, b \lor c, 
			\neg c \lor b$ & $\phi$ & Prop ($a \lor \neg b$) & 2 \\
			$\bullet a \neg b$ & $a \lor \neg b, \neg a \lor \neg b, 
			b \lor c, \neg c \lor b$ & $\phi$ & Prop $(b \lor c)$ & 3 \\
			$\bullet a \neg b\ c$ & $a \lor \neg b, \neg a \lor \neg b, 
			b \lor c, \neg c \lor b$ & $\phi$ & Confl $(\neg c \lor b)$ & 4 \\
			$\bullet a \neg b\ c$ & $a \lor \neg b, \neg a \lor \neg b, b \lor c, 
			\neg c \lor b$ & $\neg c \lor b$ & Expl $(b \lor c)$ & 5 \\
			$\bullet a \neg b\ c$ & $a \lor \neg b, \neg a \lor \neg b, 
			b \lor c, \neg c \lor b$ & $b$ & Expl $(\neg a \lor \neg b)$ & 6 \\
			$\bullet a \neg b\ c$ & $a \lor \neg b, \neg a \lor \neg b, 
			b \lor c, \neg c \lor b$ & $\neg a$ & Learn $(\neg a)$ & 7\\
			$\bullet a \neg b\ c$ & $a \lor \neg b, \neg a \lor \neg b, 
			b \lor c, \neg c \lor b, \neg a$ & $\neg a$ & Backj $(\neg a)$ & 8 \\
			$\neg a$ & $a \lor \neg b, \neg a \lor \neg b, b \lor c, 
			\neg c \lor b, \neg a$ & $\phi$ & Prop $(a \lor \neg b)$ & 9 \\
			$\neg a \neg b$ & $a \lor \neg b, \neg a \lor \neg b, b \lor c, 
			\neg c \lor b, \neg a$ & $\phi$ & Prop $(b \lor c)$ & 10 \\
			$\neg a \neg b\ c$ & $a \lor \neg b, \neg a \lor \neg b, b \lor c, 
			\neg c \lor b, \neg a$ & $\phi$ & Confl $(\neg c \lor b)$ & 11 \\
			$\neg a \neg b\ c$ & $a \lor \neg b, \neg a \lor \neg b, b \lor c, 
			\neg c \lor b, \neg a$ & $\neg c \lor b$ & Expl $(b \lor c)$ & 12 \\
			$\neg a \neg b\ c$ & $a \lor \neg b, \neg a \lor \neg b, b \lor c, 
			\neg c \lor b, \neg a$ & $b$ & Expl $(a \lor \neg b)$ & 13\\
			$\neg a \neg b\ c$ & $a \lor \neg b, \neg a \lor \neg b, b \lor c, 
			\neg c \lor b, \neg a$ & $a$ & Expl $(\neg a)$ & 14 \\
			$\neg a \neg b\ c$ & $a \lor \neg b, \neg a \lor \neg b, b \lor c, 
			\neg c \lor b, \neg a$ & $\bot$ & Fail & 15 \\
		\end{tabular}
	\end{center}
	\caption{Example for Propositional level DPLL(T)}
	\label{fig:propex}
\end{figure}

Consider the following example from~\cite{DBLP:conf/fmcad/KatzBTRH16}.
The formula F expressed as a set in CNF is:
\begin{center}
$\{a \lor \neg b, \neg a \lor \neg b, b \lor c, \neg c \lor b\}$
\end{center}
Figure~\ref{fig:propex} shows how DPLL solves this 
formula. Notice that at Step 11, the $Fail$ rule can be 
applied to conclude that the formula is unsatisfiable, and 
this is what the solver would typically do. However, for 
producing a proof of unsatisfiability, it is necessary for the 
conflict clause to be the empty clause $\bot$, and hence the 
solver explains on the conflict instead (explained in 
section \ref{sec:proofs}). A few steps later, 
the solver does conclude that the formula is unsatisfiable, 
while deriving the most basic conflict, the empty clause.


\subsubsection{Theory Rules}
		\label{sec:theoryrules}

\begin{figure}[t]
	\centering
	\begin{framed}
		\begin{tabular}{c}
			
			$\infer[Prop_i]{M := Ml}	{l \in Lit_F \cup Int_M & 
				\models_i l_1 \lor ... \lor 
				l_n \lor l & \neg l_1, ..., \neg l_n \in M & l,\neg l \notin M}$ \\ \\
			$\infer[Confl_i]{C := \{l_1 \lor ... \lor l_n \}}
			{C = \phi & \models_i l_1 \lor ... \lor l_n & 
				\neg l_1, ..., \neg l_n \in M}$ \\ \\
			$\infer[Expl_i]{C := \{l_1 \lor ... \lor l_n \lor D\}}
			{C = \{\neg l \lor D\} & \models_i l_1 \lor ... \lor l_n \lor l & 
				\neg\ l_1, ..., \neg\ l_n \prec_M l}$ \\ \\
			$\infer[Learn_i]{F := F \cup \{l_1[\textbf{c}] 
				\lor ... \lor l_n[\textbf{c}]\}}
			{l_1,...,l_n \in Lit_{M|i} \cup Int_M \cup L_i & 
				\models_i \exists \textbf{x}(l_1[\textbf{x}] \lor ... 
				\lor l_n[\textbf{x}])}$ \\ \\
		\end{tabular}
	\end{framed}
	\caption{Transition rules at Theory level. $\textbf{x}$ is a possibly empty 
		tuple of variables, and $\textbf{c}$ is a tuple of fresh constants 
		from $C$ - the set of shared constants between the sorts - of the
		same sort as $\textbf{x}$; $L_i$ is a finite set consisting of literals
		not present in the original formula $F$.}
	\label{fig:theoryrules}
\end{figure}

The rules in Figure~\ref{fig:theoryrules} model the 
interaction between the SAT solver and the theory solvers.
These rules maintain the invariant that every 
conflict clause and learned 
clause is entailed in T by the initial clause set.
The rules are analogous to their propositional rules, but 
have reasoning powers at the level of theories. For example, 
$x \land y$ is true at the propositional level, if both 
$x$ and $y$ are true. However, if the variables store these 
arithmetic theory literals: \\
$x = (a > 3)$ and $y = (a < 0)$, then $x \land y$ is 
unsatisfiable in the theory of arithmetic. So even though 
the $Conflict$ rule wont recognize this inconsistency, 
the $Conflict_{LIA}$ rule can recognize this, as long as 
the arithmetic solver is able to generate this fact as a 
$theory\ lemma$, which is a fact that is true in the theory, 
generated in CNF form by the theory solvers. This example
as a lemma would look like this:
$\neg (a > 3) \lor \neg (a < 0)$ which would trigger the 
$Conflict_i$ clause given that $x$ and $y$ above are assigned 
to true, to satisfy $x \land y$. Thus, the $Propagate_i$, 
$Conflict_i$, $Explain_i$, and $Learn_i$ rule are similar 
to their propositional versions except that each rule is 
associated to a theory $i$ and operates on lemmas 
provided by that theory. 


\subsubsection{DPLL(T) Example}
\label{sec:thoeryex}

\begin{figure}[t]
	\begin{center}
		\begin{tabular}{l l l l l}
			\textbf{M} & \textbf{F} & \textbf{C} & \textbf{Rule} 
			& \textbf{Step}\\
			\hline
			& $1, 2 \lor 3, 4, \neg 5, 6$ & $\phi$ & 
			$Prop^+$ & 1 \\
			$1\ 4\neg5\ 6$ & $1, 2 \lor 3, 4, \neg 5, 6$ & 
			$\phi$ & $Prop_{1}(\neg 1 \lor 2)$ & 2 \\ 
			$1\ 4\neg5\ 6\ 2$ & $1, 2 \lor 3, 4, \neg 5, 6$ & 
			$\phi$ & $Confl_{1}(\neg 2 \lor 5)$ & 3 \\
			$1\ 4\neg5\ 6\ 2$ & $1, 2 \lor 3, 4, \neg 5, 6$ & 
			$\neg 2 \lor 5$ & $Expl_{1}(\neg 1 \lor 2)$ & 4 \\	
			$1\ 4\neg5\ 6\ 2$ & $1, 2 \lor 3, 4, \neg 5, 6$ & 
			$\neg 1 \lor 5$ & $Expl(1)$ & 5 \\
			$1\ 4\neg5\ 6\ 2$ & $1, 2 \lor 3, 4, \neg 5, 6$ & 
			$5$ & $Expl(\neg 5)$ & 6 \\
			$1\ 4\neg5\ 6\ 2$ & $1, 2 \lor 3, 4, \neg 5, 6$ & 
			$\bot$ & $Fail$ & 7 \\
		\end{tabular}
	\end{center}
	\caption{Example for Propositional level DPLL(T)}
	\label{fig:theoryex}
\end{figure}

Consider the following example that consists of terms 
in the theory of equality over uninterpreted 
functions (EUF). Since this is the only theory we consider 
$T$ the to be the combination of $T_1 = EUF$.
The input formula for the solver 
in CNF is: \\
$(a = b) \land (f(a) = f(b) \lor f(g(a)) = h(b)) \land
(f(a) = c) \land (c \neq d) \land (f(b) = d)$ \\
Abstracting the terms, we get: \\
$1 \land (2 \lor 3) \land 4 \land \neg 5 \land 6$ \\
DPLL(T)'s execution on the example is shown in 
Figure\ref{fig:theoryex}. The interesting steps here 
are steps 2 to 4 that use theory lemmas. The clauses 
$\neg 1 \lor 2$ and $\neg 2 \lor 5$ are theory lemmas.
Intuitively, they are saying "if 1 is true, then 2 is 
true" (by congruence), and "if 2 is true, then 5 is true"
(from the information gained from 2,4, and 6), respectively. 
Essentially, these clauses are valid in the theory of 
EUF, and the solver conveniently gives us these lemmas 
to help us solve the problem at the propositional level.
As in the propositional example, we could've used 
the Fail rule after the conflict to conclude that 
the formula is unsatisfiable (Step 4), but we want 
to reduce the conflict clause to the empty disjunction
$\bot$ since it will help us with our proof.


\section{DPLL(T) Proofs}
\label{sec:proofs}
This section describes the general framework used for a 
DPLL(T) solver to produce proofs for unsatisfiable formulas. 
A proof for a satisfiable formula is just a model that 
satisfies that formula, that is an assignment of values to 
the variables in the formula that make the formula evaluate 
to $True$. For an unsatisfiable formula, the proof involves 
transforming formula into a simple contradiction by means 
of proof rules or inference rules. 

An important rule is that of resolution, which is introduced 
in Section~\ref{sec:res}. Proofs of unsatisfiability are then 
explained, at the propositional level in 
Section~\ref{sec:propproofs}, and at the theory level in 
Section~\ref{sec:theoryproofs}.

\subsection{Propositional Resolution}
\label{sec:res}
Logical calculi operate by means of rules of inferences. A
rule of inference consists of a number of premises and a 
conclusion. The rule of inference specifies a schema for a 
logical calculus where, if the premises are true, then 
the conclusions are true. For example, \\
$\infer[modus\ ponens]{\psi}{\phi \Rightarrow \psi & \phi}$ \\
Modus ponens is a rule in classical logic that says that 
"if A implies B is true $and$ if A is true, then B is true".

The inference rule that is heavily used in construction of 
proofs for SMT solvers is propositional resolution. It is
stated as follows. \\
$\infer[resolution]{\phi_1 \lor ... \lor \phi_n \lor \psi_1 \lor ... \lor \psi_m}
{\phi_1 \lor ... \lor \phi_n \lor \chi & \neg \chi \lor \psi_1 \lor ... \lor \psi_m}$ \\

Assuming the usual interpretations of the conjunction
($\land$) and disjunction ($\lor$) in classical logic, the 
intuition behind the resolution rule is explained using the 
following instance of the resolution rule. \\
$\infer{a \lor c}{a \lor \neg b & b \lor c}$ \\
If the first premise is true, and $b$ is true - that is, 
$\neg b$ is false, then $a$ must be true. If the second 
premise is true, and $b$ is false, then $c$ must be true.
Now, both premises must be true for the conclusion to be true, 
and $b$ must be either true or false in classical logic. 
So, either $a$ must be true, or $c$ must be true, and this 
is the logical representation of our conclusion.

There exist logical calculi - that is sets of inference 
rules - that, given a set of clauses, can generate all 
clauses that are logically entailed by this set. In other 
words, these calculi are $generatively\ complete$. Resolution 
is a powerful rule since it alone constitutes a logical 
calculus. Even though the resolution calculus 
isn't generatively complete, it has the useful property that 
if a set of clauses are unsatisfiable, then the calculus will 
derive the empty clause $\bot$ from this set of clauses. 


\subsection{Proofs at the Propositional Level}
\label{sec:propproofs}

\begin{figure}[t]
	\begin{center}
		\begin{tabular}{l l l l l}
			\textbf{M} & \textbf{F} & \textbf{C} & \textbf{Rule} & \textbf{Step}\\
			\hline
			& $a \lor \neg b, \neg a \lor \neg b, b \lor c, \neg c \lor b$ 
			& $\phi$ & Dec & 1 \\
			$\bullet a$ & $a \lor \neg b, \neg a \lor \neg b, b \lor c, 
			\neg c \lor b$ & $\phi$ & Prop ($a \lor \neg b$) & 2 \\
			$\bullet a \neg b$ & $a \lor \neg b, \neg a \lor \neg b, 
			b \lor c, \neg c \lor b$ & $\phi$ & Prop $(b \lor c)$ & 3 \\
			$\bullet a \neg b\ c$ & $a \lor \neg b, \neg a \lor \neg b, 
			b \lor c, \neg c \lor b$ & $\phi$ & Confl $(\neg c \lor b)$ & 4 \\
			$\bullet a \neg b\ c$ & $a \lor \neg b, \neg a \lor \neg b, b \lor c, 
			\neg c \lor b$ & $\neg c \lor b$ & Expl $(b \lor c)$ & 5 \\
			$\bullet a \neg b\ c$ & $a \lor \neg b, \neg a \lor \neg b, 
			b \lor c, \neg c \lor b$ & $b$ & Expl $(\neg a \lor \neg b)$ & 6 \\
			$\bullet a \neg b\ c$ & $a \lor \neg b, \neg a \lor \neg b, 
			b \lor c, \neg c \lor b$ & $\neg a$ & Learn $(\neg a)$ & 7\\
			$\bullet a \neg b\ c$ & $a \lor \neg b, \neg a \lor \neg b, 
			b \lor c, \neg c \lor b, \neg a$ & $\neg a$ & Backj $(\neg a)$ & 8 \\
			$\neg a$ & $a \lor \neg b, \neg a \lor \neg b, b \lor c, 
			\neg c \lor b, \neg a$ & $\phi$ & Prop $(a \lor \neg b)$ & 9 \\
			$\neg a \neg b$ & $a \lor \neg b, \neg a \lor \neg b, b \lor c, 
			\neg c \lor b, \neg a$ & $\phi$ & Prop $(b \lor c)$ & 10 \\
			$\neg a \neg b\ c$ & $a \lor \neg b, \neg a \lor \neg b, b \lor c, 
			\neg c \lor b, \neg a$ & $\phi$ & Confl $(\neg c \lor b)$ & 11 \\
			$\neg a \neg b\ c$ & $a \lor \neg b, \neg a \lor \neg b, b \lor c, 
			\neg c \lor b, \neg a$ & $\neg c \lor b$ & Expl $(b \lor c)$ & 12 \\
			$\neg a \neg b\ c$ & $a \lor \neg b, \neg a \lor \neg b, b \lor c, 
			\neg c \lor b, \neg a$ & $b$ & Expl $(a \lor \neg b)$ & 13\\
			$\neg a \neg b\ c$ & $a \lor \neg b, \neg a \lor \neg b, b \lor c, 
			\neg c \lor b, \neg a$ & $a$ & Expl $(\neg a)$ & 14 \\
			$\neg a \neg b\ c$ & $a \lor \neg b, \neg a \lor \neg b, b \lor c, 
			\neg c \lor b, \neg a$ & $\bot$ & Fail & 15 \\
		\end{tabular}
	\end{center}
	
	\begin{prooftree}
		\AxiomC{$\neg c \lor b$ (4)}
		\AxiomC{$b \lor c$ (5)}
		\BinaryInfC{$b$}
		\AxiomC{$\neg a \lor \neg b$ (6)}
		\BinaryInfC{$\neg a$ (14)}
		\AxiomC{$\neg c \lor b$ (11)}
		\AxiomC{$b \lor c$ (12)}
		\BinaryInfC{$b$}
		\AxiomC{$a \lor \neg b$ (13)}
		\BinaryInfC{$a$}
		\BinaryInfC{$\bot$}
	\end{prooftree}
	\caption{Proof tree for example in Figure~\ref{fig:propex}}
	\label{fig:propproof}
\end{figure}

At the propositional level, the proof system works as follows.
Given an execution of an DPLL(T) based SMT solver that ends 
in the $fail$ state, we can prove that the input formula 
$F_0$ is unsatisfiable as follows. We use the resolution 
calculus mentioned above to build a refutation tree - 
a tree with the input clauses and the learned clauses at the 
leaves, that lead - through the application of the resolution
rule - to the empty clause $\bot$. The learned clauses aren't 
technically in the leaf position, since they need to be 
justified as well. A learned clause is obtained, when an input 
clause that is conflicted by the context $M$ is modified by 
the application of the $Explain$ rule. In fact, the $Explain$ 
rule resolves the current conflict clause with a clause 
belonging to $F$ to obtain a new conflict clause. So proofs 
for learned clauses are also resolution proofs that are 
built by observing the conflict clauses and the explanations 
applied.  

Figure~\ref{fig:propproof} shows the proof tree for the 
unsatisfiability of F from the example in Figure~\ref{fig:propex}, 
reproduced for convenience. The nodes of the tree contain 
the clause being used followed by the step number from 
algorithm in parentheses. The left subtree is a proof of
$\neg a$. The right subtree is the proof of $a$ which is 
a learned clause.

Steps 1 to 7 constitute the left subtree of the proof tree that
adds $\neg a$ as a learned clause to F
(Although $\neg a$ is actually used in Step 14, 
it is derived in Step 7). Steps 8 to 14 constitute 
the left subtree of the proof tree.
Conflict clauses and explanations represent propositional resolution
between clauses and the proof tree has a leaf for every application 
of the Conflict and the Explain rules, and the node at the root 
represents the Fail rule. The $\neg a$ node at the root of the 
left subtree
is an exception to this (it represents an explanation, but isn't a leaf
in the tree), but notice that it is a learned clause; 
so the tree could be considered as a split of the current tree 
at the $\neg a$ node. The left subtree of the new tree 
would then just consist of $\neg a$ as a leaf, and the proof 
for $\neg a$ would be a satellite proof tree representing 
the left sub-tree of the current tree. That is how it is depicted 
in ~\cite{DBLP:conf/fmcad/KatzBTRH16}.

\subsection{Proofs at the Theory Level}
\label{sec:theoryproofs}
\begin{figure}[t]
\begin{center}
	\begin{tabular}{l l l l l}
		\textbf{M} & \textbf{F} & \textbf{C} & \textbf{Rule} 
		& \textbf{Step}\\
		\hline
		& $1, 2 \lor 3, 4, \neg 5, 6$ & $\phi$ & 
		$Prop^+$ & 1 \\
		$1\ 4\neg5\ 6$ & $1, 2 \lor 3, 4, \neg 5, 6$ & 
		$\phi$ & $Prop_{1}(\neg 1 \lor 2)$ & 2 \\ 
		$1\ 4\neg5\ 6\ 2$ & $1, 2 \lor 3, 4, \neg 5, 6$ & 
		$\phi$ & $Confl_{1}(\neg 2 \lor 5)$ & 3 \\
		$1\ 4\neg5\ 6\ 2$ & $1, 2 \lor 3, 4, \neg 5, 6$ & 
		$\neg 2 \lor 5$ & $Expl_{1}(\neg 1 \lor 2)$ & 4 \\	
		$1\ 4\neg5\ 6\ 2$ & $1, 2 \lor 3, 4, \neg 5, 6$ & 
		$\neg 1 \lor 5$ & $Expl(1)$ & 5 \\
		$1\ 4\neg5\ 6\ 2$ & $1, 2 \lor 3, 4, \neg 5, 6$ & 
		$5$ & $Expl(\neg 5)$ & 6 \\
		$1\ 4\neg5\ 6\ 2$ & $1, 2 \lor 3, 4, \neg 5, 6$ & 
		$\bot$ & $Fail$ & 7 \\
	\end{tabular}
\end{center}
	
	\begin{prooftree}
		\AxiomC{$\neg 5$ (6)}
		\AxiomC{$EUF\ Proof$}
		\UnaryInfC{$\neg 2 \lor 5$ (3)}
		\AxiomC{$EUF\ Proof$}
		\UnaryInfC{$\neg 1 \lor 2$ (4)}
		\BinaryInfC{$\neg 1 \lor 5$}
		\AxiomC{$1$ (5)}
		\BinaryInfC{$5$}
		\BinaryInfC{$\bot$}
	\end{prooftree}
	\caption{Proof tree for example in Figure~\ref{fig:theoryex}}
	\label{fig:thoeryproof}
\end{figure}

At the propositional level, we saw proof trees as being 
refutation trees that derived $\bot$ at the rooting using
input clauses and learned clauses at the leaves, with 
learned clauses supported by satellite resolution proofs. 
At the theory level, we have clauses called $theory\ lemmas$
that the theory solvers present to us to help us 
apply constraints at the theory level. Since the theory solvers 
come up with the lemmas, the proof is also expected to be 
at the theory level. And thus the theory solver also 
provides us with the proof for the lemma. In our refutation tree,
when we arrive at a leaf with a theory clause, we ask the 
theory solver for its proof and plug it into the tree.

Figure~\ref{fig:thoeryproof} shows the proof tree for 
Figure~\ref{fig:theoryex}, reproduced in the figure. The part here
that is different from the propositional case here is the theory 
lemmas in steps 3 and 4. Notice these aren't leaves in the proof 
tree. A proof from the theory solver is plugged in to support 
these lemmas. The parenthesized numbers indicate the 
relevant step from the algorithm.

\subsection{Theory Solver Proofs}
	\label{sec:theorysolverproofs}
So far, section~\ref{sec:proofs} has presented the 
proof-producing system of SMT solvers as building 
refutation trees up to the root containing $\bot$, 
from leaves that are either inputs or learned clauses 
with supported proofs. In section~\ref{sec:theoryproofs}
we added to this picture, the idea of proofs from 
theory solvers. This section talks about these proofs 
produced by theory solvers which aren't the resolution 
proofs we have got used to seeing. Instead, these 
proofs are in natural deduction~\cite{NatDed}, which 
is a calculus that is more expressive than the 
resolution calculus, and can even be generatively 
complete. The main rule we are concerned with is 
the $proof\ by\ contradiction$ rule, which works as 
follows. If you assume $\psi$ and derive $False$, 
then $\neg \psi$ is true. Theory solvers use 
theory specific rules along with primarily the 
proof by contradiction rule, to prove their lemmas.

Consider the following example 
from~\cite{DBLP:conf/fmcad/KatzBTRH16}.
The EUF proof for the lemma 
$L = (x \neq y) \lor (z \neq f(y)) \lor (f(x) = z)$ is 
as follows. \\
1. Convert the lemma into its negation: \\
$\neg L : \neg ((x \neq y) \lor (z \neq f(y)) \lor (f(x) = z))$ \\
which is equivalent to \\
$L_1 : (x = y) \land (z = f(y)) \land (f(x) \neq z)$. \\ \\

2. Prove that $L_1$ is false using rules of EUF as necessary. 
\begin{prooftree}
	\AxiomC{$f(x) \neq z$}
	\AxiomC{$x = y$}
	\RightLabel{Cong.}
	\UnaryInfC{$f(x) = f(y)$}
	\AxiomC{$z = f(y)$}
	\RightLabel{Symm.}
	\UnaryInfC{$f(y) = z$}
	\RightLabel{Trans.}
	\BinaryInfC{$f(x) = z$}
	\BinaryInfC{$\bot$}
\end{prooftree}
The leaves of this contain input formulas, or the literals
in $L_1$. The congruence rule (Cong.) gives $f(a) = f(b)$, given 
$a = b$. The other two rules refer to symmetry (Symm.) and 
transitivity (Trans.) of equality. 


\section{Comparison}
\label{sec:comp}
This section compares the proof systems of 3 state-of-the-art 
solvers CVC4, Z3, and VeriT as described in~\cite{DBLP:conf/fmcad/KatzBTRH16},~\cite{DBLP:conf/lpar/MouraB08}, 
and~\cite{DBLP:conf/tacas/FontaineMMNT06} respectively.

Although each paper introduces proof production in an SMT 
solver, they are from different times, and aimed at different 
goals. This section begins by summarizing each paper and then 
comparing the proof production systems by some particular 
metrics.

\subsection{CVC4}
\label{sec:cvc4}
The proof production system of the CVC4 SMT solver is described 
in \cite{DBLP:conf/fmcad/KatzBTRH16}. The paper introduces the 
concept of lazy proof production for SMT solvers. In DPLL(T) 
the SAT solver reasons about the input formula using feedback 
from theory solvers in the form of theory lemmas. The proof 
of unsatisfiability is a refutation proofs that has input clauses,
learned clauses, and theory lemmas at its leaves, 
and derives the empty clause at the root. The SAT solver is able 
to build the refutation proofs for the input formula; however,
these proofs are completed using proofs for the theory
lemmas from the theory solver. 

One way to produce these proofs would be to have the theory solver 
produce proofs eagerly every time a theory lemma is generated. 
However, in \cite{DBLP:conf/fmcad/KatzBTRH16}, they found that
doing this lazily - that is, all proofs for theory lemmas are 
generated only after the final refutation tree has been found. 
Once the tree is generated, the solver asks the respective theory
solver for the proofs for each theory lemma to complete the tree.

This means that each theory lemma occurring in the proof gets 
processed twice: once when they are generated for solving, and 
again when they need to be proved for the tree. However, by 
producing proofs lazily, in most cases the solvers save a lot 
of computation since many lemmas produced during solving do not
end up contributing to the final refutation.

\cite{DBLP:conf/fmcad/KatzBTRH16} discusses lazy proof 
production for three different theories - equality with
uninterpreted functions (EUF), arrays with extensionality 
(AX), and bitvectors (BV). For EUF, they propose a 
completely lazy approach. For AX, they do lazy proof 
productions with some book-keeping. Briefly, arrays are 
representable as terms in the AX theory - $a[i]$ is 
the result of reading the value at index $i$ in array $a$
and $a[i] := b$ is the result of writing value $b$ to 
array $a$ at index $i$. An axiom that the array solver 
uses to produce proofs is the extensionality axiom:
for any two arrays $a$ and $b$, i $a \neq b$ then 
there exists a $k$ such that $a[k] \neq b[k]$.
Since this axiom requires that the disequality of two 
arrays is witnessed by an index ($k$), the value of
$k$ needs to be stored so that when the proof is 
being produced for this unsatisfiability lazily, the 
solver can use this information to produce the proof.
Finally, proof production for BV, elaborated in 
~\cite{DBLP:conf/lpar/HadareanBRTD15}, is done 
semi-lazily. Bit-vectors are arrays of binary bits, 
so they are solved by a technique called $bit-blasting$
where bitvector formulas are converted into an 
equisatisfiable propositonal formula which is solved 
by an internal SAT solver. Since redoing the 
bit-blasting technique can be expensive, the SAT 
solver eagerly records a trace of the bit-blasting 
refutations. In contrast, the proof for the 
bit-blasting process - converting a formula 
to a propositional one, is done lazily for lemmas 
that occur in the final proof. 

The paper concludes by showing comparisons between 
lazy and eager proofs for EUF and AX- 
they extend CVC4 with lazy and eager proof production
systems, and check the proofs for a set of 
standard benchmarks. They produce proofs in a 
format called LFSC (Logical Framework with Side Conditions)
~\cite{DBLP:journals/fmsd/StumpORHT13} that has its own
proof checker. The evaluations show that for most 
of the benchmarks, CVC4 produces lazy proofs faster 
than eager ones.

\subsection{Z3}
\label{sec:z3}
Z3 is an SMT solver developed at Microsoft Research. 
~\cite{DBLP:conf/lpar/MouraB08} elaborates on the proof 
production system of Z3, highlighting their approach of 
$implicit quotation$ and their natural deduction style 
proofs for theory lemmas. Implicit quotation is an 
implementation detail that saves the Z3 solver 
computation when it comes to CNF conversion of input
formulas of the SMT solvers. As explained in 
\ref{sec:cnf}, a formula that needs to be checked 
for satisfiability by an SMT solver is first converted 
to conjunctive normal form or CNF. This is done by 
recursively taking the subterms of the input formula, 
and creating equivalences between the subterms and 
fresh variables. Since an equivalence is logically 
the same as a conjunction of inferences, this method 
fits well into CNF conversion. For example, 
$a \iff b$ is converted to CNF as follows. \\
$Step\ 1:\ a\ \iff b$ \\
$Step\ 2:\ (a \Rightarrow b) \land (b \Rightarrow a)$ \\
$Step\ 3:\ (\neg a \lor b) \land (\neg b \lor a)$ \\
Introducing these fresh variables increases the 
number of literals in the formula. Z3 treats the 
subterm itself as a literal. To distinguish the 
literal representation from the sub-term, they quote
it. If $a$ above is a fresh literal, and b is a 
sub-term representing theh formula $x \land y$,
instead of introducing the fresh literal $a$, 
Z3 stores a quoted version of the formula - 
$\lceil x \land y \rceil$. By doing this, they 
avoid all the extra clauses that are added to the 
formula to maintain this equivalence, as 
shown in the example above. Instead, the name of the 
literal gives enough information to tell us 
what formula it is abstracting. They provide an 
example of using implicit quotation for slack
variables during linear integer arithmetic solving.

This paper also explains the proof producing 
calculus, which is similar to the calculus explained in
sections \ref{sec:prelim} to \ref{sec:proofs}, 
except that they choose a different representation. 

They also emphasize their natural deduction style 
proofs as opposed to resolution proofs that were already
in use by SAT solvers at the time. These are mainly
proofs by contradiction for theory lemmas, which 
is also discussed in the other papers and discussed in 
Section\textcolor{red}{LEMMA PROOFS}. Their main rules
can be summarized as follows. A rule called 
$unit_resolution$ handles a general form of resolution.
The $hypothesis$ rule states a statement that needs to 
be proved. This is done by the $lemma$ rule by 
a proof by contradiction.

An important characteristic of proofs in Z3 are that the
proof producer axiomatizes many theory rewrites. These 
can be considered holes in proofs that must be proven 
by the proof checker. As argued by
~\cite{DBLP:conf/fmcad/KatzBTRH16}, this gives the proof 
checker work that is more nontrivial than just checking 
proofs - they actually need to fill in parts of the proofs. 
~\cite{DBLP:conf/lpar/MouraB08} argues that this is a 
trade-off that saves the solver time in producing proofs 
and that proof checkers are fairly well-equipped to 
prove these axioms.

\subsection{VeriT}
\label{sec:verit}
\cite{DBLP:conf/tacas/FontaineMMNT06} describes the proof
production sytem of the haRVey SMT solver, which was 
precursor to the solver VeriT. This paper is more 
application-oriented than the other two. As mentioned in 
Section \ref{sec:intro}, verification tools can be 
evaluated using three different metrics - soundness or 
trustability, automation, and expressiveness. There are 
classes of tools that score highly on one or two of those 
qualities, but not all. The metrics themselves are conflicting
in practice. Higher automation is achieved by smart heuristics
and algorithms coded into a system that is continuously 
updated with optimizations. This results in an increasingly 
large codebase that is difficult to keep bug-free as is 
the case with SMT solvers. Soundness is achieved by 
restricting the kernel to a small number of 
axioms and inference rules and applying these 
rules in a more interactive way when necessary. 
Expressive logics such as higher order logic and 
set theory allow the user to be more precise about 
their specification. However, these logics are much
harder to deal with in terms of solving. The 
state-of-the-art consists of tools that are either high 
on the expressiveness and the soundness scores, or
on the automation score. 
\cite{DBLP:conf/tacas/FontaineMMNT06} combines harVey, 
a highly automated tool with Isabelle/HOL a $proof assistant$, 
which is a tool that interactively allows a user 
to prove theorems in higher-order logic. The user
benefits from the expressive higher-order logic and 
the high trustability of Isabelle/HOL while also 
benefiting from the automation of the SMT solver.
The trade-off of course is the work done to have 
the SMT solver produce proofs of its work to the 
proof assistant, before its results are trusted, which has 
been the theme in all the papers that I chose to study.
This is an improvement on previous work that use 
the SMT solver as an oracle, without requiring it to 
produce proofs.

HaRVey provides $proof hints$ to Isabelle that 
helps the proof assistant reconstruct the proofs 
of the theorems proven by the SMT solver. Proof 
assistants such as Isabelle offer users functions 
known as $tactics$ that help the user in their 
proof aspirations. HaRVey benefits from converting 
its proofs into expressions in $sequent calculus$, 
which is an alternate notation for formulas in 
natural deduction. Modulo this encoding as sequents, 
proof production is similar to the methods described 
in this report. 

They extended Isabelle with the $sat$ and $satx$ 
that can be called in the proof assistant for 
proving subgoals of the current goal being proven.
 
This work also suggests optimization techniques 
for SMT solving such as operating on $partial 
models$ that are models of a formula where not all 
literals in the formula are assigned a value. This has 
the benefit that finding a partial model that is 
unsatisfiable eliminates multiple full models that the 
partial model can be extended to. For instance, if 
a formula contains literals $a = b$ and 
$f(a) \neq f(b)$ and a 1000 other literals, all models 
in which $a = b$ and $f(a) \neq f(b)$ are unsatisfiable, 
regardless of the values of the 1000 other literals.

Finally, the paper presents a $congruence closure$ 
algorithm to solve formulas in the theory of 
equality over uninterpreted functions (EUF). In 
this theory, we only know the function symbols 
as they appear in terms, without knowing their 
full behavior. The algorithm propagates the equalities
from the formula using the rules of reflexivity, 
transitivity, and symmetry along with the congruence rule
that gives $f(a) = f(b)$ given that $a = b$. Once a 
disequality is found between terms found to be equal, 
the algorithm concludes that the formula is unsatisfiable.

The implementation interface between Isabelle and haRVey 
takes proofs for lemmas in the theory of EUF and 
reconstructs them for the proof assistant. In summary,
this is how the interface works. When the user is trying 
to prove something in Isabelle, they may have the 
subgoal of proving a formula $F$, that is to show that 
$F$ is valid. This is the same as checking whether 
$\neg F$ is unsatisfiable. So the user invokes the $sat$ 
or $satx$ tactic from Isabelle which has haRVey check 
$\neg F$ for unsatisfiability. If $\neg F$ is 
satisfiable, the model provided by haRVey is given 
to the user as a counterexample for $F$. 
If it is unsatisfiable, then haRVey constructs a proof
trace of the unsatisfiability along with proofs of 
any theory lemmas used, encodes them as sequents,
and sends it to the interface at Isabelle, 
where the proof is reconstructed.

\subsection{Differences and Similarities}
All three papers discuss the proof production systems of 
DPLL(T)-based SMT-solvers. Sections \ref{sec:prelim} to 
\ref{sec:proofs} essentially explain the concepts that 
these systems have in common - the DPLL(T) algorithm and 
resolution-based proofs. Each paper does take a different
approach in explaining this concept. This report borrows 
the formalizations of these concepts from 
~\cite{DBLP:conf/fmcad/KatzBTRH16} which seemed to 
do the best job of elaboration with examples.

\textit{Eager vs Lazy Proof Production.} CVC4 introduces 
the idea of lazy proof production in the paper, and argues 
for their use by presenting evaluations on particular 
fragments of the solver. The authors also suggest that one 
approach might work better than the other, depending on 
the theory solver concerned. Z3 takes an eager approach, 
in which it skip logging the steps taken by the solver.
Instead it produces proofs during conflict resolution.
 
\textit{Fine-Grained vs Coarse-Grained Proofs.} CVC4 
produces fine-grained proofs in which the tinier
details of the proofs are also accounted for. 
Input formulas are converted 
to CNF; the CNF clauses, the learned clauses, and the 
theory lemmas constitute the leaves of the proof tree.
While the input clauses are axioms, the learned clauses 
are entailed by the input formula and this entailment is
proven by a separate resolution tree fro each clause. 
Theory clauses are produced by theory solvers which 
also provide the proofs for these clauses. The advantage 
of fine-grained proofs is that they are complete, and 
the proof checker only needs to check them. The disadvantage
is that the proofs must be well specified by the solver.
There may be more rules, and since none of the 
truths in the theories are assumed, they must be specified 
as well. Proofs in Z3 are more coarse-grained in that they
have holes in them that the proof assistant must fill in
during proof checking/reconstruction. These holes might be 
obvious to fill for certain theories and prove practical 
in such cases. VeriT, or haRVey in this case, suggests 
taking a slightly coarse-grained approach as well, since 
they produce proof hints to the proof assistant to 
help it reconstruct the proof. However, this was aimed 
at a particular application, and papers published by 
the VeriT group seem to have fine-grained proofs 
~\cite{DBLP:conf/cade/BarbosaBF17}.

\textit{Rewriting.} Formula rewriting is an aspect 
of SMT solving that wasn't mentioned in this work 
so far. Apart from conversion to a normal form, 
theory solves rewrite simple truths in the formula.
For example, $x + 0$ could be rewritten to $x$. 
Because of how solvers work to combine multiple 
theory solvers, these rewrites might complicate 
things by changing what a formula appears like. In CVC4's
lazy solvers, it is even more dangerous because 
solving and proving are decoupled. So a formula may 
look one way while solving, and be reduced to a 
rewritten form before the proof is extracted. The 
solution for this is to store the information of the 
rewrite so that the proof producer can prove the 
formula before it was rewritten, and the formula 
is then rewritten with some justification provided 
for the rewrite itself. Note that in the experiments 
that they conducted, the rewrites were still 
unimplemented, and thus axiomatized. In Z3, the 
rewrites seem to be prime candidates for the holes
in the proofs - they are axiomatized and the proof
checker is expected to prove them.

\textit{Proof Format.} Although these papers don't 
talk about the formats of their proofs, these solvers
have other publications that explain in detail their 
proof format. CVC4 outputs proofs in the LFSC 
(Logical Framework with Side Conditions) format 
~\cite{DBLP:journals/fmsd/StumpORHT13} that 
was developed by many of the same authors as 
~\cite{DBLP:conf/fmcad/KatzBTRH16}. LFSC is based on 
a simply typed $\lambda$-calculus with dependent types. 
The main idea is that the type theory uses the 
Curry-Howard isomorphism or the propositionas-as-types/
programs-as-proofs analogy to reduce 
proof checking to type checking. LFSC takes an 
input proof term and a signature which declares the 
constants and functions in the theories, along 
with the proof rules. In~\cite{Besson2011AFP}, the 
developers of VeriT proposed a proof format that 
motivated by SMTLib, a standard syntax for 
SMT solvers~\cite{BarFT-SMTLIB} which it 
currently uses. To the best of my knowledge, Z3 
doesn't have a paper describing its proof format, 
but it does have its own proof format. Even though 
SMT solvers have been able to converge on a common 
input format in SMTLib, the same can't be said for 
a proof format, as explained in~\cite{Fontaine2014ProofsIS}.


Proof production in theories - proof by contradiction.
\section{Conclusion and Moving Forward}
\label{sec:conc}
In this report, I have tried to summarize my investigation
of the proof production systems of three state-of-the-art 
SMT solvers - CVC4, Z3, and VeriT. In the process,
I have learned how SMT solvers produce refutation 
proofs for the formulas that they solve as unsatisfiable, 
by combining the internal SAT solver with its theory
solvers. The papers also explain how theory solvers 
produce their proofs by means of a natural deduction
calculus. This report presents these findings after 
introducing the basic working of an SMT solver. 
It also compares the the papers of the respective 
solvers and discusses some points of difference 
such as granularity of proofs, lazy proof production, 
and handling of rewrites by the solvers.

The report also mentions the motivation for SMT 
solvers to produce proofs - these highly automatic 
tools with thousands of lines of code are ubiquitous 
in the verification community, and thus would 
benefit from giving soundness guarantees. One of 
the papers briefly investigates the connection between 
an SMT solver and a proof assistant. My motivation 
is aligned with this kind of work. The Coq proof 
assistant~\cite{coq} is a popular proof checker 
that falls in the category of trustable verification 
tools. SMTCoq~\cite{DBLP:conf/cav/EkiciMTKKRB17} is a 
Coq plugin that is able to leverage the power of an 
SMT solver to prove goals in Coq. The support 
for CVC4 in SMTCoq has been implemented for a 
few theories by my advisor, Cesare Tinelli's 
research group at the Computer Science Department 
at the University of Iowa. My goal is to extend 
this support to other theories and the literature 
survey of the proof producing systems of SMT solvers
is an important step toward that goal.

\bibliographystyle{abbrv}
\bibliography{bib}

\end{document}
